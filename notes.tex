\documentclass[a4paper,12pt]{article}

\usepackage{amsmath}
\usepackage[polish]{babel}
\usepackage[a4paper]{geometry}
\usepackage{listings}

\begin{document}
    \title{Aproksymacja rozwiązania równania potencjału elektromagnetycznego metodą elementów skończonych}
    \author{Mateusz Kosman}
    % \date{}
    \maketitle

    \allowdisplaybreaks

    \section{Równanie potencjału elektromagnetycznego}
    Dane jest równanie:
    \begin{equation}
        \frac{d^2\phi}{dx^2} = -\frac{\rho}{\epsilon_r} \label{main_equation}
    \end{equation}
    o następujących warunkach brzegowych:
    \begin{gather}
        \phi'(0)+\phi(0) = 5 \label{robin}\\
        \phi(3) = 2 \label{dirichlet}
    \end{gather}
    oraz następujących wartościach zmiennych:
    \begin{gather}
        \rho = 1 \label{rho_val}\\
        \epsilon_r =
        \begin{cases}
            10,& x \in [0,1] \\
            5,& x \in (1,2] \\
            1,& x \in (2,3]
        \end{cases} \label{epsilon_val}
    \end{gather}
    Chcemy znaleźć aproksymację rozwiązania równania (\ref{main_equation}) korzystając z podanych danych.

    \section{Wyprowadzenie sformułowania wariacyjnego}
    Równanie (\ref{main_equation}) mnożymy obustronnie przez funkcję testową \(v(x)\) spełniającą warunek \(v(3) = 0\).
    Otrzymane równanie obustronnie całkujemy po dziedzinie \(\Omega = [0,3]\) i przekształcamy równoważnie:
    \begin{align}
        \frac{d^2\phi}{dx^2}(x) \cdot v(x) &= -\frac{\rho}{\epsilon_r} \cdot v(x) \nonumber \\
        \phi'' v &= -\frac{\rho}{\epsilon_r} v \nonumber \\
        \int_{0}^{3} \phi'' v\,dx &= \int_{0}^{3} -\frac{\rho}{\epsilon_r} v\,dx \nonumber \\
        \int_{0}^{3} \phi'' v\,dx &= -\frac{\rho}{\epsilon_r} \int_{0}^{3} v\,dx \nonumber \\
        \left[\phi' v\right]_{0}^{3} - \int_{0}^{3} \phi' v'\,dx &= -\frac{\rho}{\epsilon_r} \int_{0}^{3} v\,dx \text{ (całk. przez części)} \nonumber \\
        -\phi'(0) v(0) - \int_{0}^{3} \phi' v'\,dx &= -\frac{\rho}{\epsilon_r} \int_{0}^{3} v\,dx\ \left(v(3) = 0\right) \nonumber \\
        (\phi(0)-5) v(0) - \int_{0}^{3} \phi' v'\,dx &= -\frac{\rho}{\epsilon_r} \int_{0}^{3} v\,dx \text{ (korzystamy z (\ref{robin}))} \nonumber \\
        \phi(0) v(0) - \int_{0}^{3} \phi' v'\,dx &= 5v(0) -\frac{\rho}{\epsilon_r} \int_{0}^{3} v\,dx \label{weak_form}
    \end{align}
    Wprowadźmy oznaczenia:
    \begin{align}
        B(\phi, v) &= \phi(0) v(0) - \int_{0}^{3} \phi' v'\,dx \label{b_def} \\
        L(v) &= 5v(0) -\frac{\rho}{\epsilon_r} \int_{0}^{3} v\,dx \label{l_def}
    \end{align}
    Równanie (\ref{weak_form}) możemy zapisać jako:
    \begin{equation}
        B(\phi, v) = L(v), \label{b_eq_l}
    \end{equation}
    gdzie \(B(\phi, v)\) jest funkcjonałem biliniowym, a \(L(v)\) funkcjonałem liniowym.

    \section{Zastosowanie metody Galerkina}
    Dzielimy dziedzinę na n obszarów.
    W związku z tym przyjmuję ciąg funkcji bazowych \((e_0, e_1, ..., e_n)\) zdefiniowanych następująco:
    \begin{equation}
        e_i(x) =
        \begin{cases}
            \frac{x-x_{i-1}}{x_i-x_{i-1}},& x \in (x_{i-1}, x_i) \\
            \frac{x-x_{i+1}}{x_i-x_{i+1}},& x \in (x_i, x_{i+1}) \\
            0,& \text{w przeciwnym wypadku}
        \end{cases} \label{e_def}
    \end{equation}
    Aproksymujemy rozwiązanie liniowe kombinacją funkcji bazowych \(e_i = e_i(x)\) z przesunięciem \(2e_n\) (zatem zakładamy, że \(w_n = 2\)), co wynika z (\ref{dirichlet}):
    \begin{gather}
        \phi \approx 2e_n + \sum_{i=0}^{n-1} w_i e_i \label{phi_approx}\\
        v \approx \sum_{i=0}^{n-1} v_i e_i \label{v_approx}
    \end{gather}
    Podstawiając aproksymację (\ref{phi_approx}) do równania (\ref{b_eq_l}) otrzymujemy:
    \begin{align*}
        B\left(2e_n + \sum_{i=0}^{n-1} w_i e_i, v\right) &= L(v) \\
        B(2e_n, v) + B\left(\sum_{i=0}^{n-1} w_i e_i, v\right) &= L(v) \\
        B\left(\sum_{i=0}^{n-1} w_i e_i, v\right) &= L(v)-B(2e_n, v) \\
        B\left(\sum_{i=0}^{n-1} w_i e_i, v\right) &= L(v)-B(2e_n, v) \\
        \sum_{i=0}^{n-1} w_i B(e_i, v) &= L(v)-B(2e_n, v)
    \end{align*}
    Do otrzymanego równania podstawiamy aproksymację (\ref{v_approx}):
    \begin{equation*}
        \sum_{i=0}^{n-1} w_i B\left(e_i, \sum_{j=0}^{n-1} v_j e_j\right) = L\left(\sum_{j=0}^{n-1} v_j e_j\right)-B\left(2e_n, \sum_{j=0}^{n-1} v_j e_j\right)
    \end{equation*}
    Skoro \(v\) jest dowolną funkcją, to możemy przyjąć, że:
    \begin{equation*}
        \forall j \in \{0,1,...,n-1\}: v_j = 
        \begin{cases}
            1,& x = x_j \\
            0,& x \neq x_j
        \end{cases}
    \end{equation*}
    Stąd otrzymujemy układ \(n\) równań:
    \begin{equation*}
        \begin{cases}
            \sum_{i=0}^{n-1} w_i B\left(e_i, e_0\right) = L\left(e_0\right)-B\left(2e_n, e_0\right) \\
            \sum_{i=0}^{n-1} w_i B\left(e_i, e_1\right) = L\left(e_1\right)-B\left(2e_n, e_1\right) \\
            \ldots \\
            \sum_{i=0}^{n-1} w_i B\left(e_i, e_{n-1}\right) = L\left(e_{n-1}\right)-B\left(2e_n, e_{n-1}\right)
        \end{cases}
    \end{equation*}
    Układ ten można przedstawić w postaci macierzowej:
    {\footnotesize
    \begin{equation*}
        \begin{bmatrix}
            B\left(e_0, e_0\right) & B\left(e_1, e_0\right) & \cdots & B\left(e_{n-1}, e_0\right) \\
            B\left(e_0, e_1\right) & B\left(e_1, e_1\right) & \cdots & B\left(e_{n-1}, e_1\right) \\
            \vdots & \vdots & \ddots & \vdots \\
            B\left(e_0, e_{n-1}\right) & B\left(e_1, e_{n-1}\right) & \cdots & B\left(e_{n-1}, e_{n-1}\right)
        \end{bmatrix}
        \begin{bmatrix}
            w_0 \\
            w_1 \\
            \vdots \\
            w_{n-1}
        \end{bmatrix}
        =
        \begin{bmatrix}
            L\left(e_0\right)-B\left(2e_n, e_0\right) \\
            L\left(e_1\right)-B\left(2e_n, e_1\right) \\
            \vdots \\
            L\left(e_{n-1}\right)-B\left(2e_n, e_{n-1}\right)
        \end{bmatrix}
    \end{equation*}
    }\\
    Mnożąc równanie obustronnie od lewej przez odwrotność macierzy z wartościami funkcjonału \(B\) otrzymujemy:
    {\scriptsize
    \begin{equation}
        \begin{bmatrix}
            w_0 \\
            w_1 \\
            \vdots \\
            w_{n-1}
        \end{bmatrix}
        =
        \begin{bmatrix}
            B\left(e_0, e_0\right) & B\left(e_1, e_0\right) & \cdots & B\left(e_{n-1}, e_0\right) \\
            B\left(e_0, e_1\right) & B\left(e_1, e_1\right) & \cdots & B\left(e_{n-1}, e_1\right) \\
            \vdots & \vdots & \ddots & \vdots \\
            B\left(e_0, e_{n-1}\right) & B\left(e_1, e_{n-1}\right) & \cdots & B\left(e_{n-1}, e_{n-1}\right)
        \end{bmatrix}
        ^{-1}
        \begin{bmatrix}
            L\left(e_0\right)-B\left(2e_n, e_0\right) \\
            L\left(e_1\right)-B\left(2e_n, e_1\right) \\
            \vdots \\
            L\left(e_{n-1}\right)-B\left(2e_n, e_{n-1}\right)
        \end{bmatrix} \label{matrix_solution}
    \end{equation}
    }

    \section{Obliczanie aproksymacji komputerowo}
    Obliczając całki iloczynów funkcji bazowych korzystamy z faktu, że \(e_i e_j \equiv 0\) dla \(|i-j| \geq 2\), ponieważ dla każdego \(x\) co najmniej jeden z czynników jest równy 0, co wynika z definicji (\ref{e_def}).\\
    Wyznaczamy wartości \(B(e_i, e_j)\) korzystając z równania (\ref{b_def}):
    \begin{align}
        B(e_i, e_j) &=
        \begin{cases}
            e_0(0)^2 - \int_{0}^{3} e_0'^2\,dx,& i = j = 0 \\
            e_i(0)^2 - \int_{0}^{3} e_i'^2\,dx,& i = j \neq 0 \\
            e_i(0)e_j(0) - \int_{0}^{3} e_i' e_j'\,dx,& |i-j| = 1 \\
            e_i(0)e_j(0),& |i-j| \geq 2
        \end{cases} \nonumber \\
        &=
        \begin{cases}
            e_0(0)^2 - \frac{3}{n},& i = j = 0 \\
            e_i(0)^2 - \frac{6}{n},& i = j \neq 0 \\
            e_i(0)e_j(0) + \frac{3}{n},& |i-j| = 1 \\
            e_i(0)e_j(0),& |i-j| \geq 2
        \end{cases} \nonumber \\
        &=
        \begin{cases}
            1-\frac{3}{n},& i = j = 0 \\
            -\frac{6}{n},& i = j \neq 0 \\
            \frac{3}{n},& |i-j| = 1 \\
            0,& |i-j| \geq 2
        \end{cases} \label{bij_val}
    \end{align}
    Z kolei wartości \(B(2e_n, e_j)\) wynoszą:
    \begin{align}
        B(2e_n, e_j) &=
        \begin{cases}
            2e_n(0)e_{n-1}(0) - \int_{0}^{3} (2e_n)' e_{n-1}'\,dx,& j = n-1 \\
            2e_n(0)e_j(0) - \int_{0}^{3} (2e_n)' e_j'\,dx,& j \neq n-1
        \end{cases} \nonumber \\
        &=
        \begin{cases}
            2e_n(0)e_{n-1}(0) + \frac{6}{n},& j = n-1 \\
            2e_n(0)e_j(0),& j \neq n-1
        \end{cases} \nonumber \\
        &=
        \begin{cases}
            \frac{6}{n},& j = n-1 \\
            0,& j \neq n-1
        \end{cases} \label{bnj_val}
    \end{align}
    Wyznaczamy wartości \(L(e_j)\) korzystając z równania (\ref{l_def}):
    \begin{align}
        L(e_j) &= 5e_j(0) - \frac{\rho}{\epsilon_r} \int_{0}^{3} e_j\,dx \nonumber \\
        &=
        \begin{cases}
            5 - \frac{3\rho}{2n\epsilon_r},& j = 0 \\
            -\frac{3\rho}{n\epsilon_r},& j \neq 0
        \end{cases} \label{lj_val}
    \end{align}
    Podstawiając wartości (\ref{bij_val}), (\ref{bnj_val}), (\ref{lj_val}) do równania (\ref{matrix_solution}) otrzymujemy:
    \begin{equation}
        \begin{bmatrix}
            w_0 \\
            w_1 \\
            w_2 \\
            w_3 \\
            w_4 \\
            \vdots \\
            w_{n-1}
        \end{bmatrix}
        =
        \begin{bmatrix}
            1-\frac{3}{n} &  \frac{3}{n} &  0 &  0 &  0 & \cdots &  0 \\
            \frac{3}{n} & -\frac{6}{n} &  \frac{3}{n} &  0 &  0 & \cdots &  0 \\
            0 &  \frac{3}{n} & -\frac{6}{n} &  \frac{3}{n} &  0 & \cdots &  0 \\
            0 &  0 &  \frac{3}{n} & -\frac{6}{n} &  \frac{3}{n} & \cdots &  0 \\
            0 &  0 &  0 &  \frac{3}{n} & -\frac{6}{n} & \cdots &  0 \\
            \vdots & \vdots & \vdots & \vdots & \vdots & \ddots & \vdots \\
            0 &  0 &  0 &  0 &  0 & \cdots & -\frac{6}{n}
        \end{bmatrix}
        ^{-1}
        \begin{bmatrix}
            5 - \frac{3\rho}{2n\epsilon_r} \\
            -\frac{3\rho}{n\epsilon_r} \\
            -\frac{3\rho}{n\epsilon_r} \\
            -\frac{3\rho}{n\epsilon_r} \\
            \vdots \\
            -\frac{3\rho}{n\epsilon_r} \\
            -\frac{3\rho}{n\epsilon_r}-\frac{6}{n}
        \end{bmatrix} \label{matrix_nums}
    \end{equation}
    Przy aproksymacji (\ref{phi_approx}) założyliśmy, że \(w_n = 2\). Stąd możemy już obliczyć wszystkie wartości ciągu \((w_0, w_1, \ldots, w_n)\) i zaproksymować \(\phi\) zgodnie z (\ref{phi_approx}):
    \begin{equation}
        \phi \approx 2e_n + \sum_{i=0}^{n-1} w_i e_i = \sum_{i=0}^{n} w_i e_i
    \end{equation}

    \section{Szczegóły techniczne}
    W ramach aproksymacji zadanego równania całki funkcji bazowych liczone są numerycznie, w związku z czym wartości obliczone w poprzedniej sekcji nie są wymagane.
    Program jest napisany w języku \textit{Python 3.12.3} i wykorzystuje biblioteki:
    \begin{itemize}
        \item \textit{matplotlib 3.6.3} - do wizualizacji danych,
        \item \textit{numpy 1.26.4} - do rachunku macierzowego.
    \end{itemize}
    Wartość \(n\) przekazywana jest przez użytkownika jako parametr uruchomieniowy aplikacji, tj. wykonywany jest kod \lstinline{N = int(sys.argv[1])}.
\end{document}